% !TeX spellcheck = it_IT
\section{Introduzione}

Si "contrappone" all'informatica applicata, ovvero qualsiasi applicazione dell'informatica atta a raggiunger uno scopo, dove l'informatica è solamente lo strumento per raggiungere in maniera efficace un obiettivo.\\
Con "\textit{informatica teorica}" l'oggetto è l'informatica stessa, si studiano i fondamenti della disciplina in modo rigoroso e scientifico. Può essere fatto ponendosi delle questioni fondamentali: il \textit{cosa} e il \textit{come} dell'informatica, ovvero cosa è in grado di fare l'informatica e come è in grado di farlo.

\paragraph{Cosa:} L'informatica è "la disciplina che studia l'informazione e la sua elaborazione automatica", quindi l'oggetto sono l'informazione e i dispositivi di calcolo per gestirla; scienza dell'informazione. Diventa lo studio come risolvere automaticamente un problema. Ma tutti i problemi sono risolvibili in maniera automatica? Cosa è in grado di fare l'informatica? \\

La branca dell'informatica teorica che studia cosa è risolvibile si chiama \textbf{Teoria della Calcolabilità}, studia cosa è calcolabile per via automatica. Spoiler: non tutti i problemi sono risolvibili per via automatica, e non potranno mai esserlo per limiti dell'informatica stessa. Cerchiamo una caratterizzazione generale di cosa è calcolabile e cosa no, si vogliono fornire strumenti per capire ciò che è calcolabile. La caratterizzazione deve essere fatta matematicamente, in quanto il rigore e la tecnica matematica permettono di trarre conclusioni sull'informatica.

\paragraph{Come:} Una volta individuati i problemi calcolabili, come possiamo calcolarli? Il dominio della \textbf{Teoria della Complessità} vuole descrivere le risoluzione dei problemi tramite mezzi automatici in termini di risorse computazionali necessarie. Una "risorsa computazionale" è qualsiasi cosa che viene consumata durante l'esecuzione per risolvere il problema, come possono essere elettricità o numero di processori, generalmente i parametri più importanti considerati sono tempo e spazio di memoria. Bisognerà definire in modo preciso cosa si intende con "tempo" e "spazio". Una volta fissati i parametri bisogna definire anche cosa si intende con "risolvere efficientemente" un problema, in termini di tempo e spazio.\\

La teoria della calcolabilità dice quali problemi sono calcolabili, la teoria della complessità dice, all'interno dei problemi calcolabili, quali sono risolvibili efficientemente.

\newpage

\subsection{Ripasso di Matematica}

\paragraph{Funzione:} Una funzione $f$ dall'insieme $A$ all'insieme $B$ è una legge che dice come associare a ogni elemento di $A$ un elemento di $B$. Si scrive
$$ f: A \rightarrow B $$
E chiamiamo $A$ dominio e $B$ codominio. Per dire come agisce su un elemento si usa $f(a) = b$, $b$ è l'immagine di $a$ secondo $f$ (di conseguenza $a$ è la controimmagine).\\
Per definizione di funzione, è possibile che elementi del codominio siano raggiungibili da più elementi del dominio, ma non il contrario. Possiamo classificare le funzioni in base a questa caratteristica:
\begin{itemize}
	\item \textbf{Iniettiva:} $f: A \rightarrow B$ è iniettiva sse $\forall a,b \in A$, $a \neq b \implies f(a) \neq f(b)$
	\item \textbf{Suriettiva:} $f: A \rightarrow B$ è suriettiva sse $\forall b \in B$, $\exists a \in A: f(a) = b$: un altro modo per definirla è tramite l'insieme immagine di $f$, definito come
	$$ \text{Im}_f = \{b \in B: \exists a, f(a) = b \} = \{f(a): a \in A \} $$
	Solitamente $\text{Im}_f \subseteq B$, ma $f$ è suriettiva sse $\text{Im}_f = B$;
	\item \textbf{Biettiva:} $f: A \rightarrow B$ è biettiva sse è sia iniettiva che suriettiva, ovvero
	$$
	\begin{array}{c l}
		\forall a, b \in A, a \neq b: & f(a) \neq f(b) \\
		\forall b \in B, \exists a \in A: & f(a) = b
	\end{array}
	\implies \forall b \in B, \exists! a \in A: f(a) = b
	$$
\end{itemize}

\paragraph{Inversa:} Per le funzioni biettive si può naturalmente associare il concetto di "inversa": dato $f: A \rightarrow B$ biettiva, si definisce inversa la funzione $f^{-1}: B \rightarrow A$ tale che $f^{-1} (b) = a \Leftrightarrow f(a) = b$.\\

\paragraph{Composizione di funzioni:} Date $f: A \rightarrow B$ e $g: B \rightarrow C$, $f$ composto $g$ è la funzione $g \circ f: A \rightarrow C$ definita come $g \circ f(a) = g(f(a))$. Generalmente non commutativo, $f \circ g \neq g \circ f$, ma è associativo.\\

\paragraph{Funzione identità:} Dato l'insieme $A$, la funzione identità su $A$ è la funzione $i_A: A \rightarrow A$ tale che $i_A (a) = a$, $\forall a \in A$.\\

Un'altra possibile definizione per l'inversa diventa:
$$ f^{-1} \circ f = i_A \wedge f \circ f^{-1} = i_B $$

%End L1