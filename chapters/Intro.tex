% !TeX spellcheck = it_IT
% !TeX root = ../it.tex

\chapter*{Introduzione}
\addcontentsline{toc}{chapter}{Introduzione}

Si "contrappone" all'informatica applicata, ovvero qualsiasi applicazione dell'informatica atta a raggiungere uno scopo, dove l'informatica è solamente lo strumento per raggiungere in maniera efficace un obiettivo. Con "\textit{informatica teorica}" l'oggetto è l'informatica stessa, si studiano i fondamenti della disciplina in modo rigoroso e scientifico. Può essere fatto ponendosi delle questioni fondamentali: il \textit{cosa} e il \textit{come} dell'informatica, ovvero cosa è in grado di fare l'informatica e come è in grado di farlo.

\paragraph{Cosa:} L'informatica è "la disciplina che studia l'informazione e la sua elaborazione automatica", quindi l'oggetto sono l'informazione e i dispositivi di calcolo per gestirla; scienza dell'informazione. Diventa lo studio come risolvere automaticamente un problema. Ma tutti i problemi sono risolvibili in maniera automatica? Cosa è in grado di fare l'informatica?

La branca dell'informatica teorica che studia cosa è risolvibile si chiama \textbf{Teoria della Calcolabilità}, studia cosa è calcolabile per via automatica. Spoiler: non tutti i problemi sono risolvibili per via automatica, e non potranno mai esserlo per limiti dell'informatica stessa. Cerchiamo una caratterizzazione generale di cosa è calcolabile e cosa no, si vogliono fornire strumenti per capire ciò che è calcolabile. La caratterizzazione deve essere fatta matematicamente, in quanto il rigore e la tecnica matematica permettono di trarre conclusioni sull'informatica.

\paragraph{Come:} Una volta individuati i problemi calcolabili, come possiamo calcolarli? Il dominio della \textbf{Teoria della Complessità} vuole descrivere le risoluzione dei problemi tramite mezzi automatici in termini di risorse computazionali necessarie. Una "risorsa computazionale" è qualsiasi cosa che viene consumata durante l'esecuzione per risolvere il problema, come possono essere elettricità o numero di processori, generalmente i parametri più importanti considerati sono tempo e spazio di memoria. Bisognerà definire in modo preciso cosa si intende con "tempo" e "spazio". Una volta fissati i parametri bisogna definire anche cosa si intende con "risolvere efficientemente" un problema, in termini di tempo e spazio. La teoria della calcolabilità dice quali problemi sono calcolabili, la teoria della complessità dice, all'interno dei problemi calcolabili, quali sono risolvibili efficientemente.

%%% Local Variables:
%%% TeX-master: "../it.tex"
%%% End:
