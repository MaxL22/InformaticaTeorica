% !TeX spellcheck = it_IT
% !TeX root = ../it.tex

\chapter{Teoria della Calcolabilità}

\section{Notazione}

\subsection{Funzioni}

\paragraph{Funzione:} Una funzione $f$ dall'insieme $A$ all'insieme $B$ è una legge che dice come associare a ogni elemento di $A$ un elemento di $B$. Si scrive
$$ f: A \rightarrow B $$
E chiamiamo $A$ dominio e $B$ codominio. Per dire come agisce su un elemento si usa $f(a) = b$, $b$ è l'immagine di $a$ secondo $f$ (di conseguenza $a$ è la controimmagine).\\
Per definizione di funzione, è possibile che elementi del codominio siano raggiungibili da più elementi del dominio, ma non il contrario. Possiamo classificare le funzioni in base a questa caratteristica:
\begin{itemize}
	\item \textbf{Iniettiva:} $f: A \rightarrow B$ è iniettiva sse $\forall a,b \in A$, $a \neq b \implies f(a) \neq f(b)$
	\item \textbf{Suriettiva:} $f: A \rightarrow B$ è suriettiva sse $\forall b \in B$, $\exists a \in A: f(a) = b$: un altro modo per definirla è tramite l'insieme immagine di $f$, definito come
	$$ \ImSet_f = \{b \in B: \exists a, f(a) = b \} = \{f(a): a \in A \} $$
	Solitamente $\text{Im}_f \subseteq B$, ma $f$ è suriettiva sse $ \ImSet_f = B$;
	\item \textbf{Biettiva:} $f: A \rightarrow B$ è biettiva sse è sia iniettiva che suriettiva, ovvero
	$$
	\begin{array}{c l}
		\forall a, b \in A, a \neq b: & f(a) \neq f(b) \\
		\forall b \in B, \exists a \in A: & f(a) = b
	\end{array}
	\implies \forall b \in B, \exists! a \in A: f(a) = b
	$$
\end{itemize}

\paragraph{Inversa:} Per le funzioni biettive si può naturalmente associare il concetto di "inversa": dato $f: A \rightarrow B$ biettiva, si definisce inversa la funzione $f^{-1}: B \rightarrow A$ tale che $f^{-1} (b) = a \Leftrightarrow f(a) = b$.\\

\paragraph{Composizione di funzioni:} Date $f: A \rightarrow B$ e $g: B \rightarrow C$, $f$ composto $g$ è la funzione $g \circ f: A \rightarrow C$ definita come $g \circ f(a) = g(f(a))$. Generalmente non commutativo, $f \circ g \neq g \circ f$, ma è associativo.\\

\paragraph{Funzione identità:} Dato l'insieme $A$, la funzione identità su $A$ è la funzione $i_A: A \rightarrow A$ tale che $i_A (a) = a$, $\forall a \in A$.\\

Un'altra possibile definizione per l'inversa diventa:
$$ f^{-1} \circ f = i_A \wedge f \circ f^{-1} = i_B $$

\paragraph{Funzioni Parziali:} Se una funzione $f: A \rightarrow B$ è definita per $a \in A$ si indica con $f(a) \downarrow$ e da questo proviene la categorizzazione: una funzione è \textbf{totale} se definita $\forall a \in A$, \textbf{parziale} altrimenti (definita solo per qualche elemento di $A$).\\

\paragraph{Insieme Dominio:} Chiamiamo \textbf{dominio} (o campo di esistenza) di $f$ l'insieme
$$ \Dom_f = \left\{a \in A | f(a) \downarrow \right\} \subseteq A $$
Quindi se $\Dom_f = A$ la funzione è totale, se $\Dom_f \subsetneq A$ allora è una funzione parziale.\\

\paragraph{Totalizzazione:} Si può \textbf{totalizzare una funzione parziale} $f$ definendo una funzione a tratti $\overline{f}: A \rightarrow B \cup \{\bot\}$ tale che
$$ 
\overline{f} (a) = \begin{cases}
	f(a) & a \in \Dom_f(a) \\
	\bot & \text{altrimenti}
\end{cases}
$$
Dove $\bot$ è il \textbf{simbolo di indefinito}, per tutti i valori per cui la funzione di partenza $f$ non è definita. Da qui in poi $B_\bot$ significa $B \cup \{\bot\}$.\\

\paragraph{Insieme delle funzioni:} L'insieme di tutte le funzioni che vanno da $A$ a $B$ si denota con
$$ B^A = \{f: A \rightarrow B \} $$
La notazione viene usata in quanto la cardinalità di $B^A$ è esattamente $|B|^{|A|}$, con $A$ e $B$ insiemi finiti.\\
Volendo includere anche tutte le funzioni parziali: 
$$ B^A_\bot = \{f: A \rightarrow B_\bot \} $$
Le due definizioni coincidono, $B^A = B^A_\bot$, ma quest'ultima permette di mettere in evidenza che tutte le funzioni presenti sono totali o totalizzate.\\ 

\subsection{Prodotto Cartesiano}

Chiamiamo \textbf{prodotto cartesiano} l'insieme 
$$ A \times B = \{(a,b) | a \in A \wedge b \in B \} $$
Rappresenta l'insieme di tutte le coppie ordinate di valori in $A$ e $B$. In generale non è commutativo, a meno che $A=B$.\\

Può essere esteso a $n$-uple di valori:
$$ A_1 \times \dots \times A_n = \{(a_1, \dots, a_n) | a_i \in A_i\} $$
Il prodotto di $n$ volte lo stesso insieme verrà, per comodità, indicato come
$$ A \times \dots \times A = A^n $$

\paragraph{Proiettore:} Operazione "opposta", il proiettore $i$-esimo è una funzione che estrae l'$i$-esimo elemento di una tupla, quindi è una funzione
$$ \pi_i: A_1 \times \dots \times A_n \rightarrow A_i \tc \pi_i (a_1, \dots, a_n) = a_i $$
La proiezione sull'asse in cui sono presenti i valori dell'insieme $a_i$.\\

\subsection{Funzione di Valutazione}
Dati $A,B$ e $B^A_\bot$ si definisce \textbf{funzione di valutazione} la funzione
$$ \omega: \bat \times A \rightarrow B \tc \omega (f,a) = f(a) $$
Prende una funzione $f$ e la valuta su un elemento $a$ del dominio. Si possono fare due tipi di analisi su questa funzione: 
\begin{itemize}
	\item Fisso $a$ e provo tutte le $f$, ottenendo un \textit{benchmark} di tutte le funzioni su $a$
	\item Fisso $f$ e provo tutte le $a$ del dominio, ottenendo il \textit{grafico} di $f$
\end{itemize}

\section{Sistemi di Calcolo}

Vogliamo modellare teoricamente un \textbf{sistema di calcolo}; quest'ultimo può essere visto come una black box che prende in input un programma $P$, dei dati $x$ e calcola il risultato $y$ di $P$ su input $x$. La macchina restituisce $y$ se è riuscita a calcolare un risultato, $\bot$ (indefinito) se è entrata in un loop.
\begin{center}
	\begin{tikzpicture}[>=stealth, auto, node distance=2cm]
		\node [block] (C) { Calcolatore };
		\node [left=of C.west, below] (P)  {$P$};
		\node [left=of C.west, above] (x)  {$x$};
		\node [right=of C] (out)  {$y/\bot$};
		
		\draw [->] (x) -- node {} (C.169);
		\draw [->] (P) -- node {} (C.194);
		\draw [->] (C) -- node {} (out);
	\end{tikzpicture}
\end{center}

Quindi, formalmente, possiamo definire un sistema di calcolo come una funzione 
$$ \C: \prog \times \dati \rightarrow \dati_\bot $$

Possiamo vedere un sistema di calcolo come una funzione di valutazione:
\begin{itemize}
	\item i dati $x$ corrispondono all'input $a$
	\item il programma $P$ corrisponde alla funzione $f$
\end{itemize}

Formalmente, un programma $P \in \prog$ è una sequenza di regole che trasformano un dato input in uno di output, ovvero l'espressione di una funzione secondo una sintassi 
$$ P: \dati \rightarrow \dati_\bot $$
e di conseguenza $P \in \dati^{\dati}_\bot$. In questo modo abbiamo mappato l'insieme $\prog$ sull'insieme delle funzioni, il che ci permette di definire il sistema di calcolo come la funzione
$$ \C: \dati^{\dati}_\bot \times \dati \rightarrow \dati $$

Analoga alla funzione di valutazione. Con $\C(P,x)$ indichiamo la funzione calcolata da $P$ su $x$ dal sistema di calcolo $\C$, che viene detta \textbf{semantica}, ovvero il suo "significato" su input $x$.\\

Il modello solitamente considerato quando si parla di calcolatori è quello di \textbf{Von Neumann}.\\

\section{Potenza Computazionale}
Indicando con 
$$ \C (P, \_): \dati \rightarrow \dati $$
la funzione che viene calcolata dal programma $P$ (semantica di $P$).\\

La \textbf{potenza computazionale} di un calcolatore è definita come l'insieme di tutte le funzioni che quel sistema di calcolo è in grado di calcolare, ovvero
$$ F(\C) = \{\C (P, \_) | P \in \prog\} \subseteq \dati_\bot^{\dati} $$

Ovvero, l'insieme di tutte le possibili semantiche di funzioni calcolabili con il sistema $\C$. Stabilire il carattere di quest'ultima inclusione equivale a stabilire \textit{cosa può fare l'informatica}:
\begin{itemize}
	\item se $F(\C) \subsetneq \dati_\bot^{\dati}$ allora esistono compiti \textbf{non automatizzabili}
	\item se $F(\C) = \dati_\bot^{\dati}$ allora l'informatica \textit{può fare tutto}
\end{itemize}

Calcolare funzioni vuol dire risolvere problemi \textit{in generale}, a ogni problema è possibile associare una funzione soluzione che permette di risolverlo automaticamente.\\

Un possibile approccio per risolvere l'inclusione è tramite la \textbf{cardinalità} (funzione che associa ogni insieme al numero di elementi che contiene) dei due insiemi. Potrebbe però presentare dei problemi: è efficace solo quando si parla di insiemi finiti. Ad esempio, l'insieme dei numeri naturali contiene l'insieme dei numeri pari $\mathbb{P} \subsetneq \mathbb{N}$, ma $|\mathbb{N}| = |\mathbb{P}| = \infty$.\\
Serve una diversa definizione di cardinalità che considera l'esistenza di infiniti \textit{più densi di altri}.\\

\section{Relazioni di Equivalenza}
Dati due insiemi $A,B$, una relazione binaria $R$ è un sottoinsieme $R \subseteq A \times B$ di coppie ordinate. Data $R \subseteq A^2$, due elementi sono in relazione sse $(a,b) \in R$. Indichiamo la relazione tra due elementi anche con la notazione infissa $aRb$. \\

Una classe importante di relazioni è quella delle \textbf{relazioni di equivalenza}: una relazione $R \subseteq A^2$ è una relazione di equivalenza sse rispetta le proprietà di
\begin{itemize}
	\item riflessività: $\forall a \in A$, $(a,a) \in R$
	\item simmetria: $\forall a,b \in A$, $(a,b) \in R \Leftrightarrow (b,a) \in R$
	\item transitività: $\forall a,b,c \in A$, $(a,b) \in R \wedge (b,c) \in R \implies (a,c) \in R$
\end{itemize}

\subsection{Partizione indotta dalla relazione di equivalenza}
A ogni relazione di equivalenza $R \subseteq A^2$ si può associare una \textbf{partizione}, ovvero un insieme di sottoinsiemi $A_i \subseteq A$ tali che
\begin{itemize}
	\item $\forall i \in \mathbb{N}^+$, $A_i \neq \emptyset$
	\item $\forall i,j \in \mathbb{N}^+$, se $i \neq j$ allora $A_i \cap A_j = \emptyset$
	\item $\bigcup_{i \in \mathbb{N}^+} A_i = A$
\end{itemize}

La relazione $R$ definita su $A^2$ \textit{induce} una partizione $\{A_1, A_2, \dots\}$ su $A$.\\

\subsection{Classi di equivalenza e Insieme quoziente}
Dato un elemento $a \in A$, chiamiamo \textbf{classe di equivalenza} di $a$ l'insieme
$$ [a]_R = \{b \in A | (a,b) \in R \} $$
Ovvero, tutti gli elementi in relazione con $a$, chiamato \textbf{rappresentante} della classe. \\

Si può dimostrare che
\begin{itemize}
	\item non esistono classi di equivalenza vuote, per riflessività
	\item dati $a,b \in A$, allora $[a]_R \cap [b]_R = \emptyset$, oppure $[a]_R = [b]_R$, i due elementi o sono in relazione o non lo sono
	\item $\bigcup_{a \in A} [a]_R = A$
\end{itemize}

L'insieme delle classi di equivalenza, per definizione, è una partizione indotta da $R$ su $A$, detta \textbf{insieme quoziente} di $A$ rispetto ad $R$, denotato con $A / R$.\\

\section{Cardinalità}

\subsection{Isomorfismi}

Due insiemi $A$ e $B$ sono \textbf{isomorfi} (\textit{equi-numerosi}) se esiste una biezione tra essi, denotato come $A \sim B$. Chiamando $\U$ l'insieme di tutti gli insiemi, la relazione $\sim$ è $\sim \subseteq \U^2$.\\

Dimostriamo che $\sim$ è una relazione di equivalenza: 
\begin{itemize}
	\item riflessività: $A \sim A$, la biezione è data dalla funzione identità $i_A$
	\item simmetria: $A \sim B \Leftrightarrow B \sim A$, la biezione è data dalla funzione inversa
	\item transitività: $A \sim B \wedge B \sim C \implies A \sim C$, la biezione è data dalla composizione delle funzioni usate per $A \sim B$ e $B \sim C$
\end{itemize}

Dato che $\sim$ è una relazione di equivalenza, permette di partizionare l'insieme $\U$, risultando in classi di equivalenza contenenti insiemi isomorfi, ovvero con la stessa cardinalità. Possiamo quindi definire la \textbf{cardinalità} come l'insieme quoziente di $\U$ rispetto alla relazione $\sim$.\\

Questo approccio permette il \textit{confronto delle cardinalità di insiemi infiniti}, basta trovare una funzione biettiva tra i due insiemi per poter affermare che sono isomorfi.\\

\subsection{Cardinalità finita}
La prima classe di cardinalità è quella delle cardinalità finite. Definiamo la seguente famiglia di insiemi:
$$ J_n = \begin{cases}
	\emptyset & \text{ se } n = 0 \\
	\{1, \dots , n\} & \text{ se } n > 0 \\
\end{cases}$$
Un insieme $A$ ha \textbf{cardinalità finita} sse $A \sim J_n$ per qualche $n \in \mathbb{N}$; in tal caso possiamo scrivere $|A| = n$. La classe di equivalenza $[J_n]_{\sim}$ identifica tutti gli insiemi di $\U$ contenenti $n$ elementi.\\

\subsection{Cardinalità infinita}
L'altra classe di cardinalità è quella delle \textbf{cardinalità infinite}, ovvero gli insiemi non in relazione con $J_n$. Si possono dividere in \textbf{numerabili} e \textbf{non numerabili}.

\subsubsection{Insiemi numerabili}
Un insieme $A$ è numerabile sse $A \sim \mathbb{N}$, ovvero $A \in [\mathbb{N}]_\sim$. Vengono anche detti \textbf{listabili}, in quanto è possibile elencare tutti gli elementi dell'insieme $A$ tramite una funzione $f$ biettiva tra $\mathbb{N}$ e $A$; grazie ad $f$ possiamo elencare gli elementi di $A$, formando l'insieme 
$$ A = \{f(0), f(1). \dots \} $$
Ed è esaustivo, in quanto elenca tutti gli elementi di $A$.\\

Questi insiemi hanno cardinalità $\aleph_0$ (\textit{aleph}).\\

\subsubsection{Insiemi non numerabili}
Gli insiemi non numerabili sono insiemi a cardinalità infinita ma non listabili, sono "più fitti" di $\mathbb{N}$; ogni lista generata non può essere esaustiva.\\

Il più noto tra gli insiemi non numerabili è l'insieme $\mathbb{R}$ dei numeri reali.\\

\begin{theor}
	L'insieme $\mathbb{R}$ non è numerabile ($\mathbb{R} \not \sim \mathbb{N}$)
\end{theor}
\begin{proof}
	Suddividiamo la dimostrazione in 3 punti: 
	\begin{enumerate}
		\item dimostriamo che $\mathbb{R} \sim (0,1)$
		\item dimostriamo che $\mathbb{N} \not \sim (0,1)$
		\item dimostriamo che $\mathbb{R} \not \sim \mathbb{N}$
	\end{enumerate}
	
	Per dimostrare che $\mathbb{R} \sim (0,1)$ serve trovare una biezione tra $\mathbb{R}$ e $(0,1)$. Usiamo una rappresentazione grafica: 
	\begin{itemize}
		\item disegnare una semicirconferenza di raggio $1/2$, centrata in $1/2$, quindi con diametro $1$
		\item disegnare la perpendicolare al punto da mappare che interseca la circonferenza
		\item disegnare la semiretta passante per il centro $C$ e l'intersezione precedente
	\end{itemize}
	L'intersezione tra asse reale (parallela al diametro) e semiretta finale è il punto mappato. 
	
	\begin{center}
		\begin{tikzpicture}[scale=6]
			
			% Real line
			\draw (0.4,0.5) -- (1.6,0.5);
			% Draw the semicircle
			\draw[dashed] (0.75,1) arc (180:360:0.25);
			% Draw the top line segment
			\draw (0.75,1) -- (1.25,1);
			
			% Labels
			\node[above] at (0.75,1) {0};
			\node[above] at (1.25,1) {1};
			\node[above] at (1,1) {\textit{C}};
			\node[above] at (1.5,0.5) {$\mathbb{R}$};
			
			% Paths for first red ray, semicircle and first intersection
			\path [name path=rr1] (0.9,0) -- (0.9,1);
			\path [name path=semicircle] (0.75,1) arc (180:360:0.25);
			\path [name intersections={of=rr1 and semicircle, by=i1}];
			
			% First red line
			\draw[dashed,red] (0.9,1) -- (i1);
			
			% Find point beyond, draw the paths for the red ray and real, find the intersection
			\coordinate (Beyond) at ($(i1)!-1.179!(1,1)$); 
			\path [name path=rr2] (1,1) -- (Beyond);
			\path [name path=r] (0.25,0.5) -- (1.75,0.5);
			\path [name intersections={of=rr2 and r, by=i2}];
			% Draw second red ray
			\draw[dashed,red] (1,1) -- (i2);
		\end{tikzpicture}
	\end{center}
	
	Questo approccio permette di dire che $\mathbb{R}$ è isomorfo a qualsiasi segmento di lunghezza maggiore di $0$. La stessa biezione vale anche sull'intervallo chiuso $[0,1]$ (e di conseguenza qualsiasi intervallo chiuso), usando la "compattificazione" $\mathbb{R} = \mathbb{R} \cup \{\pm \infty\}$ e mappando $0$ su $-\infty$ e 1 su $+ \infty$.\\
	
	Continuiamo dimostrando che $\mathbb{N} \not \sim (0,1)$: serve dimostrare che l'intervallo $(0,1)$ non è listabile, quindi che ogni lista manca di almeno un elemento. Proviamo a "costruire" un elemento che andrà a mancare. Per assurdo, sia $\mathbb{N} \sim (0,1)$, allora possiamo listare gli elementi di $(0,1)$ come 
	$$ 
	\begin{array}{c c c c c}
		0. & a_{00} & a_{01} & a_{02} & \dots \\
		0. & a_{10} & a_{11} & a_{12} & \dots \\
		0. & a_{20} & a_{21} & a_{22} & \dots \\
		0. & \multicolumn{4}{c}{\dots}
	\end{array}
	$$
	dove con $a_{ij}$ indichiamo la cifra di posto $j$ dell'$i$-esimo elemento della lista.\\
	
	Costruiamo il numero $c = 0.c_0 c_1 \dots$ tale che
	$$ c_{i} = \begin{cases}
		2 & \text{ se } a_{ii} \neq 2 \\
		3 & \text{ se } a_{ii} = 2 \\
	\end{cases}$$
	
	Viene costruito "guardando" le cifre sulla diagonale principale, apparterrà sicuramente a $(0,1)$ ma differirà per almeno una posizione (quella sulla diagonale principale) da ogni numero presente all'interno della lista. Questo è assurdo sotto l'assunzione che $(0,1)$ è numerabile, quindi abbiamo provato che $\mathbb{N} \not \sim (0,1)$.\\
	
	Il terzo punto $\mathbb{R} \not \sim \mathbb{N}$ si dimostra per transitività.\\
	
	Più in generale, non si riesce a listare nessun segmento di lunghezza maggiore di 0.\\
\end{proof}

Questa dimostrazione (punto 2 in particolare) è detta \textbf{dimostrazione per diagonalizzazione}.\\

L'insieme $\mathbb{R}$ viene detto \textbf{insieme continuo} e tutti gli insiemi isomorfi a $\mathbb{R}$ si dicono continui a loro volta.\\

Gli insiemi continui hanno cardinalità $\aleph_1$.\\

%Decidi se sub o subsub
\subsection{Insieme delle Parti}

%P16






